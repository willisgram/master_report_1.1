%===================================== CHAP 9 =================================

\chapter{Recommendations for Further Research}

The FPLDP is not an in-depth researched area. In fact, only two articles on Fantasy football were found (\cite{Matthews}; \cite{Bonomo}) and only one of them specifically addressed the FPLDP \citep{Matthews}. To the author's knowledge, the mathematical model presented is the first of its kind for the Fantasy Premier League, and the concept of modeling gamechips have never been considered in Fantasy Sports before. Thus, the novelty of the FPLDP makes recommendations for future research topics a vital issue to address.

\newpar

When the model is run with realized points, the results demonstrate that there is a significant gap between the optimal solution and the top manager. Moreover, considering the sample size of 5.9 million participating managers, it illustrates that there is much uncertainty in the FPLDP. Thus, instead of taking the deterministic solution method adopted in the thesis, using a stochastic solution method or a method based on machine learning are ideas for further research. Modeling the FPLDP as a Markov Decision Problem is also possible, as done by \cite{Matthews}.

\newpar

A potential for improvement of the forecast-based optimization model lies in the implementation of the gamechips. As seen in the thesis, the implementation does not guarantee an increased performance when the gamechips are played. Thus, the qualitative strategies suggested do not seem complementary to the forecast-based optimization model. The Wildcard, in particular, could benefit from another form of implementation. Thus, examining the effect of different implementations is an exiting topic for further research. One interesting approach is to introduce the gamechips to the model in all sub-problems, but punish the objective function for the use of gamechips to keep it from using all gamechips in the first sub-problems. Another approach would be to implement the gamechips analogous to the exercise of a financial option. Other approaches related to the Triple Captain include using a risk-adjusted measure analogous to the Sharpe Ratio \citep{Sharpe} or an approach based on the optimal solution of the Secretary Problem (\cite{Lindley}; \cite{Bruss}).


\newpar

Multiple approaches with a potential of improving forecasts of player points exist. Some are solely related to the fact that in the future, more data will be available. For example, a parameter tuning can be conducted for the Regression method. Furthermore, with more data available, one can fit the regression model by only considering matches at a similar stage of previous seasons. For instance, total goals can have a high predictive power of realized points in the second part of the season, but perhaps not in the first couple of gameweeks. Methods for handling players without data due to for instance transfers or promotions are also of interest. These could, for instance, be based on the assumption that players of the same price are expected to perform similarly.

\newpar

Other topics for future research related to the forecasts are not founded in the scarcity of data. Instead, they are connected to discoveries made in the thesis. For instance, it is clear that the Odds method performs best at the beginning of the season. Thus, combining forecasting methods is an interesting approach to improve performance. For example, the Odds method could be used for the first 5 gameweeks before switching to the Modified Average method for the rest of the season. In combination, the model could be allowed to use the Wildcard in the gameweek forecasting methods are changed. Alternatively, odds could be used as an explanatory variable in the regression. Finally, forecasting of future price could be used as a measure to profit from trading players, thus obtaining a higher budget.

\newpar

An analysis of risk handling was presented in this thesis. The analysis indicated that the risk handling could be used to improve the performance or alternatively to reduce risk. However, further research is required before more conclusive statements can be made.