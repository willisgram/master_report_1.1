%===================================== CHAP 8 =================================
\chapter{Concluding Remarks}

This thesis describes an optimization model for the Fantasy Premier League Decision Problem by suggesting an approach for optimizing Fantasy Premier League decisions. To the author's knowledge, the mathematical model presented is the first of its kind for the Fantasy Premier League. Also, to the best of the author's knowledge, the concept of modeling gamechips have never been considered in Fantasy Sports, and risk handling has never been considered in Fantasy Premier League.

\newpar

The mathematical model has been run with realized points in order to obtain the optimal solution. Furthermore, the mathematical model has been solved by the use of a rolling horizon heuristic with forecasts of player points as input. The forecasts are generated by three different methods: the Modified Average method, the Regression method and the Odds method. Moreover, the gamechips are implemented by use of qualitative strategies. The strategies are mostly developed by considering whether a gameweek is blank or double. In addition, risk handling constraints are added and their impact is studied.

\newpar

Solving for realized points reveals a considerable potential for improvement in performance in Fantasy Premier League. Even the top manager among 5.9 million users only manages to achieve approximately 50\% of the optimal solution. This speaks in volumes of the nature of uncertainty in the FPLDP. Furthermore, it motivates researching the problem, as the top performing solutions are far from optimal. For the forecast-based optimization model we have developed, achieving a position among the top 30\% of all managers appears obtainable by all marks. Moreover, forecasts based either on the Regression or Modified Average method are deemed most promising. For these methods, a top 10\% finish appears to be within reach.

\newpar

The effect of adding risk handling constraints related to variance has been analyzed. The results are promising and indicate that a strategy that considers which players face each other in a gameweek can either be used to limit variance or boost performance, depending on a risk preference threshold. Obtaining both effects simultaneously, however, appears to be impossible. In the best cases, the model reaches approximately the top 1.5\% of managers when risk handling is introduced alongside forecasts from the Modified Average method. Winning the Fantasy Premier League, i.e., finishing as the top manager, appears to be unachievable with the methods suggested in this thesis.

\newpar

Even though the results are encouraging, there exist some drawbacks worth addressing. Most notably, the implementation of the gamechips is unsatisfactory in the sense that with forecasts from the Modified Average and Odds method, the performances are aggravated when gamechips are introduced. Another drawback is the sparsity of data. As data for only two seasons are available, the results must be interpreted carefully. 


\newpar

As 5.9 million managers compete in the Fantasy Premier League, competition is fierce. In fact, a mean of only 1.9 points per gameweek separates a top 10\% finish from a top 20\% finish in 2017/2018 season. As of now, the mathematical model developed can already be used as a decisions support tool. For one, it can be run ex-post to answer insightful questions like "What would have been the optimal transfer last gameweek?" or "What is the optimal team so far?". Furthermore, the forecast-based optimization model can be used as a complementary support tool for FPL managers. For instance, it can propose the $k$ best set of transfers for a manager's team in each round. This can be done by running the model $k$ times, in each case constraining the optimal solutions previously generated. The growing attention from commercial actors and the monetary figures seen in other Fantasy Sports, such as in Daily NFL Fantasy Sports, indicate that improving average performance by only a small amount may be of substantial worth in the future. Thus, further development of the forecast-based optimization model, as well as more research on the FPLDP in general, has great potential value.