%===================================== CHAP 1 =================================

%Only for right page counting---------------
\pagestyle{fancy}
\fancyhf{}
\renewcommand{\chaptermark}[1]{\markboth{\chaptername\ \thechapter.\ #1}{}}
\renewcommand{\sectionmark}[1]{\markright{\thesection\ #1}}
\renewcommand{\headrulewidth}{0.1ex}
\renewcommand{\footrulewidth}{0.1ex}
\fancyfoot[LE,RO]{\thepage}
\fancyhead[LE]{\leftmark}
\fancyhead[RO]{\rightmark}
\fancypagestyle{plain}{\fancyhf{}\fancyfoot[LE,RO]{\thepage}\renewcommand{\headrulewidth}{0ex}}

\pagenumbering{arabic} 				
\setcounter{page}{1}
%--------------------------------------------

\chapter{Introduction}\label{introduction}


For the past decade, a phenomenon called Fantasy Sports has experienced an enormous growth, both in terms of active participants and media attention. According to \cite{fantasysports_users}, it is estimated that 59.3 million people in the U.S and Canada are competing in some kind of Fantasy Sports. Furthermore, the Fantasy Sports industry is expected to grow annually by 41\% and generate \$14.4 billion by the year 2020 \citep{forbes}. Fantasy Sports is an online game where participants assemble imaginary teams consisting of real players within a professional sport. Fantasy teams compete in both private- and public leagues, where the teams gain points  based on players' week-by-week performances in real-life matches. Fantasy Premier League (FPL) is the Fantasy Sports of England's top division football league, Premier League. It is the largest Fantasy Sports in Europe, with more than 5.9 million active participants. 

\newpar


Typically, Fantasy Sports participants select a squad consisting of a given amount of players in each position on the field. Each real-life player is given a buying price based on their skill level and ability to perform well in the Fantasy Sports. The performance of each player in the actual games is measured in terms of points, where an action like scoring a goal will provide the player with a specified amount of points. In order to avoid participants from selecting only top rated real-life players, they are typically given a restricted budget.

\newpar


Fantasy Sports has its origin from the 1950s when a limited partner of the Oakland Raiders, Wilfred Winkenbach, invented a Fantasy Sports for golf. Each participant selected a team of professional golfers, and the one who suggested the team with the best score would win the tournament. In 1962, Winkenbach suggested the first fantasy league for American Football, initiating the birth of what has become a billion-dollar business \citep{mccracken2012culturematic}. Nowadays, Fantasy Sports are available for several sports, including American football, basketball, ice hockey, baseball, football and cricket.  


\newpar

Over the last years, the interest for Fantasy Sports has grown rapidly among commercial actors. For instance, a phenomenon called Daily Fantasy Sports (DFS) has arisen. It is a version of Fantasy Sports, allowing participants to compete for money over short time-periods, typically for one day or a weekend. In the U.S, FanDuel and DraftKings are the main providers of DFS, with FanDuel being valued to more than \$1 billion \citep{forbes_fanduel}. Both FanDuel and DraftKings were granted licenses from the United Kingdom Gambling Commission in 2015 \citep{Purdum}. Furthermore, in 2017, the Scandinavian bookmaker Unibet launched Fantasy Sports Beta. Fantasy Sports Beta allows customers to compete for money within a Fantasy Sports for Premier League. In addition, the Norwegian bookmaker GUTS included bets on Fantasy Premier League in 2016. They offer bets on whether one player will perform better than another for the upcoming week.

\newpar

In the U.S., an estimated 30\% of Fantasy Sports participants use additional websites to research athletes and other factors when they construct their fantasy teams. In total, these participants spend \$656 million annually to purchase additional information and decision-making tools \citep{fantasy_decision_tools}. There exist several optimization models for Fantasy Sports, for instance ice hockey \citep{drafting_hockey_pools} and American football \citep{Fry}. However, compared to the models suggested for the mentioned Fantasy Sports, there is a difference when modelling Fantasy Premier League. For instance, classical American Fantasy leagues consist of selecting a team for an entire season and do not allow for transfers. On the contrary, Fantasy Premier League allows for making player transfers during the season. Thus, FPL participants have to consider making new decisions each gameweek in a season. To the author's knowledge, an optimization model making optimal decisions in Fantasy Premier League has not been developed before. Given number of participants, commercial actors' and their willingness to invest both money and time in Fantasy Sports, we see a potential for developing a model optimizing decisions in the Fantasy Premier League. Sportradar, a Norwegian company providing data for international bookmakers, has assisted us in writing this thesis by providing relevant data.

\newpar

A model optimizing decisions in the FPL could serve two purposes. First, many actors may be interested in detailed analysis of the FPL run ex-post, i.e. after points are realized. Such a model could provide answers to questions like "what is the optimal team so far?" and "what are the optimal transfers for each particular gameweek?". Furthermore, the model can be used to determine what would have been the optimal transfers in a particular gameweek for a given participant's starting team. Moreover, the optimal solution can be compared to the score of the participant who finished on top in the overall rankings and the gap can be assessed. Second, the model can be run ex-ante, i.e. before points are realized. Thereby, it can be used to determine how well the model performs when competing by the same rules as FPL participants and subsequently for decision support. However, in this case, forecasts of points are required. Therefore, in this thesis, three methods of generating such forecasts are suggested. Thus, the scope of this thesis is two-folded. First, the aim is to formulate a mathematical model for the FPLDP. Secondly, the aim is to develop a framework which could compete with human participants in terms of overall ranking. 

\newpar

An additional motivation for this thesis is that the methods suggested can be generalized for decision-making similar to those in FPL. For instance, the model may be of inspiration to actual football teams when considering which players to sign during a transfer window. Even though some adjustments should be made, the characteristics of the problems are related. For instance, a football team has a restricted budget when purchasing new players, just as in FPL. Moreover, FPL participants base their transfers on projected points gained by a player, while football managers purchase players they expect to perform well in the future. By further comparison, FPL consists of a set of players to select, while the actual clubs can buy players from a similar set of players, namely the set of all players available in a transfer window.

\newpar

The thesis is structured as follows. Chapter 2 provides the problem description. Here, the rules and point system of Fantasy Premier League are presented. Chapter 3 provides a literature review. First, studies of real-life football that can be related to Fantasy Sports are presented. Then, a presentation of Fantasy Sports optimization models, player performance forecast and a comparison to other well-known problems is given. Chapter 4 describes the mathematical formulation of the problem. Chapter 5 presents the solution framework, including a description of the solution method used for solving the mathematical model. In addition, three methods for generating forecasts are described. Chapter 6 presents a computational study undertaken to set up the framework before it is run on the 2017/2018 season of the FPL. In Chapter 7, the results of the performance of the framework in 2017/2018 season are presented. In Chapter 8, the concluding remarks highlight the most important results and implications of the research. Finally, Chapter 9 addresses the potential for further research.
