\usepackage{setspace}
\usepackage{graphicx}
\usepackage{amssymb}
\usepackage{mathrsfs}
\usepackage{amsthm}
\usepackage{amsmath}
\usepackage{color}
\usepackage{booktabs}
\usepackage{float}
\usepackage{subcaption}
%\floatstyle{boxed} 
\restylefloat{figure}
\usepackage{algorithm}
%\usepackage[noend]{algpseudocode}
\usepackage{makecell} %for å få linebreak inne i en celle
\usepackage{algpseudocode}
\usepackage{enumitem} %for å få i) istedenfor a) i listing. 
\usepackage{dsfont} %for å få større greske bokstaver.
\usepackage{relsize} %for å få større greske bokstaver. 
\usepackage[multiple]{footmisc} % for å få dobbel fotnote.

\usepackage{verbatim}
\usepackage[Lenny]{fncychap}
\usepackage[pdftex,bookmarks=true]{hyperref}
\usepackage[pdftex]{hyperref}
\usepackage{dcolumn}% for å få negative tall i tabeller.
\newcolumntype{d}[1]{D{.}{.}{#1}}% for å få negative tall i tabeller.
\usepackage{siunitx} % for å få negative tall i tabeller.
\urlstyle{same}
\hypersetup{
    colorlinks,%
    citecolor=black,%
    filecolor=black,%
    linkcolor=black,%
    urlcolor=black
}

%appendix
\usepackage[toc,page]{appendix}

\usepackage[font=small,labelfont=bf]{caption}
\usepackage{fancyhdr}
\usepackage{times}
%\usepackage[intoc]{nomencl}
%\renewcommand{\nomname}{List of Abbreviations}
%\makenomenclature
\usepackage{natbib}
\usepackage{setspace}




% ShareLaTeX does not support glossaries now. Sorry...
%\usepackage[number=none]{glossary}
%\makeglossary
%\newglossarytype[abr]{abbr}{abt}{abl}
%\newglossarytype[alg]{acronyms}{acr}{acn}
%\newcommand{\abbrname}{Abbreviations} 
%\newcommand{\shortabbrname}{Abbreviations}
%%\makeabbr
\newcommand{\HRule}{\rule{\linewidth}{0.5mm}}

\renewcommand*\contentsname{Table of Contents}

\pagestyle{fancy}
\fancyhf{}
\setlength\parindent{0pt} %Slik at man ikke får noe indent på nye paragraf.
\newcommand{\newpar}{\vskip 0.5cm} %Slik at hvert nytt avsnitt begynner med ny blank linje.
\renewcommand{\chaptermark}[1]{\markboth{\chaptername\ \thechapter.\ #1}{}}
\renewcommand{\sectionmark}[1]{\markright{\thesection\ #1}}
\renewcommand{\headrulewidth}{0.1ex}
\renewcommand{\footrulewidth}{0.1ex}
\fancypagestyle{plain}{\fancyhf{}\fancyfoot[LE,RO]{\thepage}\renewcommand{\headrulewidth}{0ex}}

\DeclareMathOperator{\EX}{\mathbb{E}} %brukes i matematisk formulering
\let\origRho\rho %for å få større rho gjennom hele teksten.
\renewcommand*\rho{\mathlarger\origRho} %for å få større rho gjennom hele teksten.