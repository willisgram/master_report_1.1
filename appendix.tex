%\section*{\begin{center}{\Huge Appendix}\end{center}}


\appendix
%\renewcommand{\thesection}{\thechapter.\alph{section}}

\renewcommand{\thechapter}{\Alph{chapter}}

%\addcontentsline{toc}{chapter}{Appendix}
%$\\[0.5cm]$

\chapter{Bonus Point System}
\section{Bonus Point System}\label{A1_BPS}

The Bonus Point System used in the scoring of Fantasy Premier League is presented here. It utilizes several statistics supplied by Opta, a sport analytics company, to create a performance score for players. The players with the top three BPS in a given match receive bonus points - three points to the highest-scoring player, two to the second best and one to the third. The table is obtained from www.premierleague.com. 

\begin{table}[H]
\centering
\begin{tabular}{|l|c|}
\hline
\textbf{Action}                                                                   & \textbf{BPS} \\
\hline
Playing 1 to 60 minutes                                                  & 3   \\
Playing over 60 minutes                                                  & 6   \\
Goalkeepers and defenders scoring a goal                                 & 12  \\
Midfielders scoring a goal                                               & 18  \\
Forwards scoring a goal                                                  & 24  \\
Assists                                                                  & 9   \\
Goalkeepers and defenders keeping a clean sheet                          & 12  \\
Saving a penalty                                                         & 15  \\
Save                                                                     & 2   \\
Successful open play cross                                               & 1   \\
Creating a big chance (a chance where the receiving player should score) & 3   \\
For every 2 clearances, blocks and interceptions (total)                 & 1   \\
For every 3 recoveries                                                   & 1   \\
Key pass                                                                 & 1   \\
Successful tackle (net*)                                                 & 2   \\
Successful dribble                                                       & 1   \\
Scoring the goal that wins a match                                       & 3   \\
70 to 79\% pass completion (at least 30 passes attempted)                & 2   \\
80 to 89\% pass completion (at least 30 passes attempted)                & 4   \\
90\%+ pass completion (at least 30 passes attempted)                     & 6   \\
Conceding a penalty                                                      & -3  \\
Missing a penalty                                                        & -6  \\
Yellow card                                                              & -3  \\
Red card                                                                 & -9  \\
Own goal                                                                 & -6  \\
Missing a big chance                                                     & -3  \\
Making an error which leads to a goal                                    & -3  \\
Making an error which leads to an attempt at goal                        & -1  \\
Being tackled                                                            & -1  \\
Conceding a foul                                                         & -1  \\
Being caught offside                                                     & -1  \\
Shot off target                                                          & -1 \\
\hline
\end{tabular}
\caption{Bonus Point System.}
\end{table}



\chapter{The Elo System}
\section{The Elo System}
\label{A2_Elo_System}

The Elo system introduced in Section \ref{Strength_of_football_teams} is a rating system used for measuring the relative strength level of sport teams and individual athletes. Within football, the Elo system can be used in order to determine a club's Elo value. The great advantage of the Elo system lies in its simplicity: there is only one value per club at each point in time, the higher the better. 

\begin{comment}
Elo values can be obtained according to equation \ref{eq5.2} and \ref{eq5.4}. Clubelo.com provide historical Elo values for most of the professional football teams in the world, including teams in the English Premier League, English 1st division etc. These values are used in order to compute the relative team strength in this project, using a k-value of 20. A great advantage of the ratings provided by Clubelo is the fact that these values are modified, taking into account home field advantage, goal difference and inter-league adjustments. In addition, the Elo values from Clubelo incorporate all fixtures and not only league matches.
\end{comment}

The performance of a team is not measured absolutely, but inferred from wins, losses and draws against other teams. If team A has a Elo value of $R_A$ and team B has a Elo value of $R_B$, team A has an expected score of 
\begin{equation}\label{eq5.2}
    E_A = \frac{1}{1+10^{\frac{R_B - R_A}{400}}}
\end{equation}
when facing team B. Similarly, team B has an expected score of
\begin{equation}\label{eq5.3}
    E_B = \frac{1}{1+10^{\frac{R_A - R_B}{400}}}
\end{equation}

\newpar
When teams play each other and win or lose, they exchange points. Hence, the club's Elo value is updated once a match has been played. Using the expected scores from equation \ref{eq5.2}, team A's updated Elo value is calculated according to
\begin{equation} \label{eq5.4}
    R^{'}_A = R_A + (S_A-E_A) \times k
\end{equation}
where $S_A$ is the result (1 for win, 0.5 for draw and 0 for loss). Further, $k$ is a factor which determines the adjustments of the Elo value. A higher k-value will increase the changes in the rating and thus the Elo values suffer from more variation. For smaller k-values, more stable Elo values are created. In chess, the World Chess Federation suggest that a value of $k = 20$ should be used for players with an Elo value below 2400. 

\newpar

Similarly, team B's updated Elo value is given by
\begin{equation} \label{eq5.5}
    R^{'}_B = R_B + (S_B-E_B) \times k
\end{equation}

\newpar

In the following, an illustrative example provides how Elo values of two teams are updated once a match between them has been played:

\newpar

Assume that team A has an Elo value of 1881 and that team B has a value of 1650. If team A faces team B, team A has an expected score according to equation \ref{eq5.2}:

\begin{equation*}
    E_A = \frac{1}{1+10^{\frac{1650-1881}{400}}} = 0.791
\end{equation*}

while team B has an expected score of

\begin{equation*}
    E_B = \frac{1}{1+10^{\frac{1881-1650}{400}}} = 0.209
\end{equation*}

Further, if team A won the match, its Elo value will increase to a value according to equation \ref{eq5.4}: 
\begin{equation*}
   R^{'}_A= 1881 + (1-0.791) \times 20 = 1885
\end{equation*}

As for team B, its Elo value decreases to

\begin{equation*}
   R^{'}_B = 1650 + (0-0.209) \times 20 = 1646 
\end{equation*}

In the following, the Elo values used for 2017/2018 season is presented in form of tables.

\begin{table}[H]
\centering
\smaller
\begin{tabular}{|l|c|c|c|c|c|c|c|c|c|}
\hline
Team           & GW1  & GW2  & GW3  & GW4  & GW5  & GW6  & GW7  & GW8  & GW9   \\
\hline
Man City       & 1864 & 1873 & 1870 & 1874 & 1900 & 1906 & 1922 & 1932 & 1947  \\
Liverpool      & 1835 & 1842 & 1851 & 1862 & 1851 & 1842 & 1854 & 1850 & 1863  \\
Tottenham      & 1885 & 1894 & 1886 & 1879 & 1903 & 1892 & 1912 & 1915 & 1929  \\
Man Utd        & 1855 & 1864 & 1876 & 1880 & 1884 & 1888 & 1914 & 1917 & 1929  \\
Chelsea        & 1907 & 1895 & 1906 & 1911 & 1923 & 1916 & 1944 & 1934 & 1921  \\
Arsenal        & 1845 & 1853 & 1843 & 1832 & 1841 & 1845 & 1860 & 1863 & 1858  \\
Leicester      & 1714 & 1715 & 1721 & 1717 & 1717 & 1711 & 1713 & 1713 & 1716  \\
Burnley        & 1626 & 1646 & 1638 & 1645 & 1656 & 1659 & 1663 & 1674 & 1679  \\
Everton        & 1749 & 1760 & 1768 & 1763 & 1756 & 1735 & 1740 & 1728 & 1733  \\
Newcastle      & 1621 & 1619 & 1612 & 1625 & 1641 & 1645 & 1646 & 1650 & 1659  \\
Bournemouth    & 1651 & 1649 & 1635 & 1631 & 1632 & 1635 & 1639 & 1640 & 1644  \\
Crystal Palace & 1640 & 1620 & 1620 & 1607 & 1607 & 1596 & 1601 & 1599 & 1619  \\
Swansea        & 1647 & 1654 & 1646 & 1659 & 1653 & 1658 & 1656 & 1650 & 1661  \\
West Ham       & 1670 & 1668 & 1664 & 1651 & 1661 & 1660 & 1665 & 1671 & 1679  \\
Southampton    & 1689 & 1690 & 1697 & 1694 & 1683 & 1688 & 1691 & 1685 & 1688  \\
Watford        & 1601 & 1609 & 1627 & 1623 & 1645 & 1633 & 1652 & 1654 & 1672  \\
Brighton       & 1581 & 1580 & 1577 & 1581 & 1598 & 1589 & 1606 & 1603 & 1611  \\
Stoke          & 1660 & 1659 & 1672 & 1674 & 1683 & 1673 & 1674 & 1680 & 1683  \\
West Brom      & 1642 & 1653 & 1664 & 1662 & 1655 & 1651 & 1656 & 1654 & 1664  \\
Huddersfield   & 1471 & 1499 & 1510 & 1513 & 1514 & 1514 & 1529 & 1525 & 1611  \\
\hline
\end{tabular}
\caption{Elo values for 2017/2018 season.}
\label{tab:elo_values_gameweeks_1}
\end{table}


\begin{table}[H]
\centering
\smaller
\begin{tabular}{|l|c|c|c|c|c|c|c|c|c|}
\hline
Team           & GW10 & GW11 & GW12 & GW13 & GW14 & GW15 & GW16 & GW17 & GW18 \\
\hline
Man City       & 1947 & 1963 & 1968 & 1974 & 1964 & 1966 & 1959 & 1968 & 1972 \\
Liverpool      & 1853 & 1857 & 1865 & 1869 & 1859 & 1866 & 1883 & 1877 & 1870 \\
Tottenham      & 1933 & 1938 & 1939 & 1932 & 1914 & 1902 & 1903 & 1907 & 1909 \\
Man Utd        & 1910 & 1920 & 1913 & 1903 & 1895 & 1900 & 1918 & 1909 & 1911 \\
Chelsea        & 1921 & 1912 & 1919 & 1929 & 1919 & 1920 & 1923 & 1909 & 1912 \\
Arsenal        & 1867 & 1860 & 1854 & 1866 & 1848 & 1851 & 1844 & 1841 & 1837 \\
Leicester      & 1721 & 1729 & 1730 & 1724 & 1713 & 1724 & 1733 & 1740 & 1754 \\
Burnley        & 1673 & 1679 & 1689 & 1696 & 1680 & 1689 & 1686 & 1682 & 1698 \\
Everton        & 1708 & 1691 & 1697 & 1695 & 1653 & 1665 & 1684 & 1690 & 1698 \\
Newcastle      & 1662 & 1655 & 1646 & 1642 & 1616 & 1618 & 1619 & 1612 & 1605 \\
Bournemouth    & 1652 & 1649 & 1658 & 1666 & 1656 & 1647 & 1649 & 1650 & 1648 \\
Crystal Palace & 1611 & 1610 & 1609 & 1609 & 1607 & 1609 & 1614 & 1614 & 1621 \\
Swansea        & 1650 & 1647 & 1637 & 1629 & 1618 & 1616 & 1614 & 1621 & 1616 \\
West Ham       & 1656 & 1656 & 1649 & 1639 & 1629 & 1618 & 1620 & 1634 & 1638 \\
Southampton    & 1691 & 1691 & 1681 & 1675 & 1676 & 1675 & 1679 & 1682 & 1667 \\
Watford        & 1667 & 1657 & 1651 & 1660 & 1665 & 1659 & 1667 & 1661 & 1654 \\
Brighton       & 1628 & 1628 & 1638 & 1637 & 1624 & 1622 & 1618 & 1605 & 1603 \\
Stoke          & 1668 & 1678 & 1678 & 1678 & 1659 & 1651 & 1660 & 1656 & 1650 \\
West Brom      & 1655 & 1651 & 1642 & 1635 & 1632 & 1630 & 1631 & 1624 & 1631 \\
Huddersfield   & 1538 & 1536 & 1546 & 1537 & 1525 & 1523 & 1521 & 1534 & 1531 \\
\hline
\end{tabular}
\caption{Elo values for 2017/2018 season.}
\label{tab:elo_values_gameweeks_2}
\end{table}


\begin{table}[H]
\centering
\smaller
\begin{tabular}{|l|c|c|c|c|c|c|c|c|c|}
\hline
Team           & GW19 & GW20 & GW21 & GW22 & GW23 & GW24 & GW25 & GW26 & GW27 \\
\hline
Man City       & 1981 & 1983 & 1985 & 1978 & 1980 & 1971 & 1972 & 1974 & 1968 \\
Liverpool      & 1878 & 1878 & 1882 & 1885 & 1890 & 1899 & 1885 & 1889 & 1888 \\
Tottenham      & 1901 & 1909 & 1912 & 1912 & 1909 & 1914 & 1908 & 1918 & 1919 \\
Man Utd        & 1914 & 1911 & 1904 & 1897 & 1904 & 1907 & 1911 & 1901 & 1903 \\
Chelsea        & 1914 & 1910 & 1912 & 1916 & 1915 & 1909 & 1915 & 1889 & 1866 \\
Arsenal        & 1839 & 1839 & 1844 & 1839 & 1840 & 1828 & 1834 & 1817 & 1824 \\
Leicester      & 1732 & 1736 & 1727 & 1723 & 1729 & 1735 & 1741 & 1732 & 1729 \\
Burnley        & 1697 & 1689 & 1696 & 1693 & 1689 & 1681 & 1677 & 1677 & 1683 \\
Everton        & 1704 & 1708 & 1708 & 1699 & 1693 & 1688 & 1685 & 1693 & 1685 \\
Newcastle      & 1603 & 1614 & 1612 & 1610 & 1620 & 1619 & 1617 & 1617 & 1620 \\
Bournemouth    & 1640 & 1638 & 1636 & 1644 & 1645 & 1657 & 1659 & 1684 & 1690 \\
Crystal Palace & 1643 & 1644 & 1639 & 1646 & 1656 & 1663 & 1658 & 1660 & 1657 \\
Swansea        & 1610 & 1609 & 1605 & 1615 & 1611 & 1613 & 1627 & 1644 & 1648 \\
West Ham       & 1654 & 1643 & 1645 & 1645 & 1658 & 1670 & 1668 & 1666 & 1654 \\
Southampton    & 1665 & 1661 & 1657 & 1664 & 1654 & 1655 & 1660 & 1657 & 1666 \\
Watford        & 1632 & 1625 & 1634 & 1624 & 1622 & 1622 & 1616 & 1618 & 1641 \\
Brighton       & 1604 & 1611 & 1609 & 1611 & 1610 & 1601 & 1595 & 1599 & 1611 \\
Stoke          & 1634 & 1643 & 1642 & 1638 & 1628 & 1625 & 1632 & 1630 & 1624 \\
West Brom      & 1628 & 1619 & 1620 & 1624 & 1618 & 1627 & 1631 & 1628 & 1620 \\
Huddersfield   & 1553 & 1557 & 1558 & 1560 & 1555 & 1543 & 1536 & 1533 & 1531 \\
\hline
\end{tabular}
\caption{Elo values for 2017/2018 season.}
\label{tab:elo_values_gameweeks_3}
\end{table}


\begin{table}[H]
\centering
\smaller
\begin{tabular}{|l|c|c|c|c|c|c|c|c|}
\hline
Team           & GW28 & GW29 & GW30 & GW31 & GW32 & GW33 & GW34 & GW35 \\
\hline
Man City       & 1983 & 1993 & 1993 & 1989 & 1989 & 1979 & 1958 & 1971 \\
Liverpool      & 1913 & 1917 & 1916 & 1902 & 1906 & 1931 & 1935 & 1938 \\
Tottenham      & 1935 & 1938 & 1923 & 1922 & 1922 & 1936 & 1939 & 1920 \\
Man Utd        & 1896 & 1902 & 1904 & 1886 & 1886 & 1893 & 1905 & 1894 \\
Chelsea        & 1876 & 1870 & 1864 & 1853 & 1853 & 1846 & 1840 & 1851 \\
Arsenal        & 1828 & 1819 & 1816 & 1818 & 1818 & 1837 & 1841 & 1830 \\
Leicester      & 1732 & 1727 & 1722 & 1727 & 1727 & 1740 & 1729 & 1718 \\
Burnley        & 1682 & 1680 & 1685 & 1693 & 1693 & 1703 & 1712 & 1712 \\
Everton        & 1700 & 1692 & 1683 & 1685 & 1693 & 1692 & 1697 & 1698 \\
Newcastle      & 1641 & 1643 & 1639 & 1645 & 1645 & 1653 & 1665 & 1676 \\
Bournemouth    & 1678 & 1676 & 1676 & 1664 & 1668 & 1673 & 1671 & 1662 \\
Crystal Palace & 1656 & 1653 & 1647 & 1638 & 1648 & 1648 & 1651 & 1657 \\
Swansea        & 1662 & 1649 & 1659 & 1651 & 1651 & 1651 & 1651 & 1651 \\
West Ham       & 1670 & 1667 & 1652 & 1631 & 1631 & 1646 & 1653 & 1650 \\
Southampton    & 1667 & 1669 & 1665 & 1645 & 1645 & 1637 & 1635 & 1634 \\
Watford        & 1639 & 1647 & 1651 & 1639 & 1635 & 1638 & 1630 & 1622 \\
Brighton       & 1620 & 1633 & 1642 & 1627 & 1627 & 1621 & 1618 & 1618 \\
Stoke          & 1629 & 1634 & 1634 & 1625 & 1617 & 1616 & 1614 & 1616 \\
West Brom      & 1623 & 1612 & 1604 & 1586 & 1581 & 1577 & 1578 & 1594 \\
Huddersfield   & 1557 & 1568 & 1565 & 1559 & 1548 & 1547 & 1551 & 1560 \\
\hline
\end{tabular}
\caption{Elo values for 2017/2018 season.}
\label{tab:elo_values_gameweeks_4}
\end{table}



\section{Walk-through of calculating adjusted average points based on Elo ratings } \label{app_walk_through}


In Table \ref{tab:Elo.1617_app}, Elo values for the first gameweeks of the English Premier League 2016/17 season are listed. Note that the values are calculated ahead of each gameweek, so that the values in the GW1 column are the Elo values of the teams before gameweek 1 has been played. 

\begin{table}[H]
\centering
\begin{tabular}{|c|l|c|c|c|c|}
\cline{2-6}
\multicolumn{1}{l|}{} & \multicolumn{1}{l|}{} & \multicolumn{4}{c|}{Elo value}  \\ \cline{2-6} 
\hline
                      & Team                  & GW1    & GW2    & GW3  & GW4    \\
                      \hline
1                     & Chelsea               & 1793   & 1800   & 1798 & 1804   \\
2                     & Tottenham             & 1804   & 1804   & 1800 & 1798   \\
3                     & Man City              & 1848   & 1856   & 1858 & 1863   \\
4                     & Man Utd               & 1789   & 1797   & 1799 & 1804   \\
-                     &                       &        &        &      &        \\
-                     &                       &        &        &      &        \\
-                     &                       &        &        &      &        \\
17                    & Watford               & 1624   & 1631   & 1618 & 1612   \\
18                    & Hull                  & 1589   & 1603   & 1613 & 1608   \\
19                    & Middlesbrough         & 1595   & 1597   & 1601 & 1604   \\
20                    & Sunderland            & 1655   & 1654   & 1636 & 1641   \\
\hline
\end{tabular}
\caption{Elo values for Premier League 2016/2017.}
\label{tab:Elo.1617_app}
\end{table}

Values from Table \ref{tab:Elo.1617_app} is used in order to weight the performance of a player. In the following, an example of weighting historical performance is presented. Paul Pogba, a midfielder at Manchester United, received the following points for the first three gameweeks: 


\begin{table}[H]
\centering
\begin{tabular}{|c|l|c|c|c|c|}
\hline
\multicolumn{1}{|c|}{} & \multicolumn{4}{c|}{Realized points} \\ \hline
Opponent               & GW1        & GW2       & GW3    & GW4   \\
\hline
Chelsea                & 6          &           &        &       \\
Sunderland             &            & 9         &        &       \\
Watford                &            &           & 10     &         \\
Tottenham              &            &           &        & ?      \\
\hline
\end{tabular}
\caption{Imaginary realized points for player Paul Pogba.}
\label{tab:appendix_paul_pogba_example}
\end{table}

If one simply used his average score for the past three gameweeks in order to predict his performance for the next gameweek, one would predict Pogba to gain 8.33 points in gameweek 4. However, in this case one does not account for the fact that Manchester United were facing teams of different quality. In addition, one does not consider the fact that Pogba plays for Manchester United, a team that in general performs better than most other teams. Hence, it is necessary to weight the previous performance taking into account the opponent team and the player's team. Equation \ref{eq:adj_avg} is used to calculate the \textit{adjusted average points}. 

\newpar

In gameweek 1, Manchester United were facing Chelsea and Pogba earned 6 points. In order to account for the fact that Pogba plays for Manchester United and was facing Chelsea (a higher ranked team), the realized points in gameweek 1 is adjusted higher:

\begin{equation*}
    \textrm{Paul Pogba GW1} = 6 \cdot \frac{1793}{1789} = 6.013 
\end{equation*}

As can be observed, his modified points increase when facing an opponent that is assumed to be better than his team. On the contrary, when facing a weaker team as Sunderland in gameweek 2 and Watfordi in gameweek 3, his realized points in gameweek 2 is adjusted lower: 

\begin{equation*}
    \textrm{Paul Pogba GW2} = 9 \cdot \frac{1654}{1797} = 8.284 
\end{equation*}


\begin{equation*}
    \textrm{Paul Pogba GW3} = 10 \cdot \frac{1618}{1799} = 8.994
\end{equation*}

Hence, his adjusted average points in gameweek 4 accounted for \textit{upcoming opponent strength}, is calculated as: 

\begin{equation*}
    \frac{1}{3} \cdot (6.013 + 8.284 + 8.994) \cdot \frac{1798}{1804} \approx 7.74 
\end{equation*}
